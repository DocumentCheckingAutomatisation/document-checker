% сделано на преддипломе: 
% 1) добавлено в раздел 1.3 еще инструменты 
% 2) добавлен раздел 2.7 про телеграм бота и кли 
% 3) обновлены задачи
% 4) обновление Приложение Б
% 5) поменяла цель


\documentclass{report}
\usepackage{settings}  % подклчючение настроек документа


\begin{document}

%%%% Вставка необходимого титульника %%%%
%\include{TitlePages/title_proizvodstv_pract.pdf}  % Титульный лист дипломной работы 
%\include{TitlePages/courseTitle}  % Титульный лист курсовой работы
\includepdf[pages={1}]{TitlePages/titul_preddiplom.pdf} % Ваш титульний лист в .pdf 

%(если ваш титульный лист в формате .docx - воспользуйтесь сервисом по конвертации из docx (Word) в pdf)


\setcounter{page}{2} % начинаем нумерацию страниц
\tableofcontents  % это содержание, которое генерируется автоматически

\setcounter{chapter}{0} % установка счетчика глав
\setcounter{section}{0} % установка счетчика разделов
\setcounter{subsection}{0} % установка счетчика подразделов
\setcounter{equation}{0} % установка счетчика формул


\chapter*{ВВЕДЕНИЕ} % звездочка нужна, чтобы не было нумерации у этой главы
\addcontentsline{toc}{chapter}{ВВЕДЕНИЕ} % чтобы глава отображалась в содержании

Проект <<Система автоматизации нормоконтроля>> направлен на создание программного продукта, который позволит значительно упростить и ускорить процесс проверки оформления студенческих работ, таких как выпускные квалификационные работы, курсовые работы и отчеты по практике, на соответствие стандартам. 

Целевой аудиторией системы являются нормоконтролеры и студенты высших учебных заведений и колледжей. Для нормоконтролеров система упрощает процесс проверки, автоматически выполняя рутинные задачи проверки и освобождая время на более сложные аспекты проверки. Для студентов решение делает процесс проверки более быстрым, понятным и удобным, даёт возможность предварительно проверить свои работы перед сдачей нормоконтролёру. 

В настоящее время на рынке отсутствуют достойные, качественно и правильно работающие системы автоматизации нормоконтроля, что делает предлагаемое решение востребованным и \textbf{актуальным}.

Разработка системы ведется с использованием современных технологий, включая язык программирования Python, инструмент тестирования unittest, систему контроля версий git. Для проверки документов будет использоваться российская open-source библиотека python-docx и российская open-source библиотека dedoc, обеспечивающие качественное извлечение и анализ текстовой информации, а также регулярные выражения.
% добавить по инструментам сюда еще

%\textbf{актуальность}

{\bf Объект исследования:} процессы нормоконтроля студенческих работ в высших учебных заведениях.

{\bf Предмет исследования:} методы и технологии автоматизации процесса нормоконтроля студенческих работ с использованием программных средств.

{\bf Цель работы:} разработка прототипа системы автоматической проверки оформления студенческих работ, обеспечивающего автоматизацию рутинных операций нормоконтроля, а также предоставляющего студентам возможность предварительной самостоятельной проверки соответствия оформления установленным требованиям.

% Прошлая цель: спроектировать систему нормоконтроля студенческих работ (курсовых работ, ВКР, отчётов по практике), предусматривающую открытый доступ для студентов к предварительной проверке с последующей интерпретацией её результатов с частичной реализацией некоторых модулей предварительной проверки.

% Индивид задание на преддиплом 
% Доработка системы нормоконтроля: внедрение новых правил в модуль проверки структуры и модуль оформления для работ различных форматов (приоритетно DOCX и LATEX); добавление интерфейса для общего использования студентами программы.


\break
\break

{\bf Задачи:}
\begin{enumarabic}
\item анализ аналогичных систем и сбор требований;
\item проектирование системы;
\item разработка модулей проверки структуры и оформления документа формата LaTeX и DOCX;
%\item разработка модуля логирования;
\item разработка API;
\item разработка телеграм-бота и CLI;
\item модульное тестирование, тестирование на реальных работах студентов.
\end{enumarabic}

{\bf Теоретическая новизна исследования} в обосновании подходов к автоматизированному анализу структуры и оформления студенческих работ на основе формализованных правил, что позволяет унифицировать и систематизировать процесс нормоконтроля.

{\bf Практическая значимость исследования} заключается в создании программного продукта, который автоматизирует процесс нормоконтроля, снижает нагрузку на нормоконтролеров, повышает точность и скорость проверки, а также предоставляет студентам инструмент для самостоятельной предварительной проверки работ. 

% Еще вариант практ. значимости: Практическая значимость исследования заключается в создании программного прототипа, обеспечивающего автоматизированную проверку оформления студенческих работ на соответствие установленным требованиям. Разработанная система может быть использована для снижения нагрузки на нормоконтролёров за счёт делегирования рутинных операций, повышения точности проверки, а также предоставления студентам возможности самостоятельной предварительной оценки корректности оформления своих работ.


\setcounter{section}{0} 
\setcounter{subsection}{0}
\setcounter{equation}{0}

\chapter{ТЕОРЕТИЧЕСКИЕ ОСНОВЫ РАЗРАБОТКИ СИСТЕМЫ АВТОМАТИЧЕСКОГО НОРМОКОНТРОЛЯ}
% или
% Анализ, проектирование и выбор инструментов для разработки системы автоматического нормоконтроля
% или
% Исследование и проектирование системы автоматического нормоконтроля


\section{Анализ аналогичных систем и потребностей пользователей}
%\section{Анализ аналогичных систем }
%\section{Анализ потребностей пользователей}
%\section{Анализ нормативной базы}

Потенциальными пользователями системы автоматизации нормоконтроля студенческих работ являются две категории: студенты и нормоконтролеры Иркутского государственного университета.

Система будет предназначена для студентов, которые смогут использовать ее для проверки оформления своих научно-исследовательских работ (ВКР, курсовых, отчетов по практике). Они смогут выполнять проверки в несколько итераций: начиная с первых версий своей работы и завершая итоговым вариантом. Окончательный вариант работы, при котором система не выявила ошибок, студент должен будет отправить нормоконтролеру на утверждение. Таким образом, использование системы позволит ускорить процесс прохождения нормоконтроля, так как студентам не придется ожидать многократных проверок от нормоконтролера.

Нормоконтролеры будут применять систему для итоговой проверки работ, отправленных студентами. Их задача будет заключаться в подтверждении соответствия всех элементов оформления требованиям и утверждении работы, если она соответствует стандартам. Система автоматизирует рутинные задачи проверки и позволит нормоконтролерам сосредоточиться на более сложных аспектах анализа документа.

На основании вышеизложенного, система одновременно облегчает процесс как для студентов, так и для сотрудников университета, занимающихся проверкой оформления.

Основной потребностью потенциальных пользователей является возможность максимально точной проверки оформления, которая должна учитывать требования актуального ГОСТ и специфические стандарты университета. %Студенты и нормоконтролеры ожидают, что система будет проверять все аспекты оформления, включая шрифты, поля, нумерацию, оформление ссылок и списка литературы.
Также важным аспектом будет скорость проверки. Система должна быть способна быстро анализировать документ, чтобы пользователь мог оперативно исправлять ошибки. Результаты проверки должны быть чёткими и понятными: система должна указывать на конкретные ошибки и сопровождать их рекомендациями по исправлению. Кроме того, важна поддержка различных форматов, включая DOCX, RTF, PDF и LATEX, что сделает систему универсальной для студентов, работающих в разных средах. % подкорректровать форматы по итогу 
Потребности нормоконтролеров включают возможность управления правилами проверки в случае изменений требований. Они ожидают, что система надежно и корректно выполнит всю рутинную работу по проверке оформления, позволяя сосредоточиться на более сложных аспектах контроля документов.

Многие существующие системы проверки оформления документов не предоставляют полноценной функциональности или применяют ошибочные правила оформления. К тому же, такие системы проверяют только ограниченный набор параметров, не учитывая специфику требований конкретного университета. Часто остается неизвестным, актуальный ли ГОСТ используется для проверки. Кроме того, большинство существующих решений ориентировано на техническую, проектную или конструкторскую документацию, что делает их непригодными для проверки студенческих работ. Еще одной важной проблемой является ограничение по форматам файлов.

Все это приводит к тому, что пользователи не могут эффективно использовать имеющиеся инструменты для решения своих задач.

На рынке представлены несколько онлайн-сервисов, позволяющих проверить документ студенческую работу на соответствие требованиям ГОСТ. Эти сервисы обычно предоставляют возможность загрузить документ, автоматически провести его проверку и получить исправленный файл. Например, сервис «Оформить работу по ГОСТу онлайн бесплатно»  позволяет загрузить работу в формате doc, docx или rtf, обработать её и скачать \break результат \cite{serv1}. Дизайн данного сервиса устаревший, интерфейс не предоставляет настройки проверки (рис. \ref{fig:pic111}).

\myfigure{0.8}{serv1}{Скриншот сервиса <<Оформить работу по ГОСТу онлайн бесплатно>>}{pic111}
\break

По результатам работы этого сервиса на примере правильно оформленной курсовой работы можно сказать, что работа проверяется и оформляется, но неправильно. Сервис применяет неверные правила ГОСТ, слова или их части без объяснения причин выделяются цветом, после чего вручную необходимо чистить работу. Задать свои правила проверки нет возможности, следовательно, сервис не может следовать особым требованиям к оформлению от конкретного университета.

Другой сервис --- ДокСтандарт --- предлагает аналогичный \break функционал \cite{serv2}. Выделяется более современным, минималистичным дизайном и интерфейсом (рис. \ref{fig:pic112}).

\myfigure{0.8}{serv2}{Скриншот сервиса <<ДокСтандарт>>}{pic112}

В данном сервисе неизвестен использующийся ГОСТ, также сервис сообщает, что работа проверена, но не дает возможности ее скачать. К тому же, нельзя настраивать правила проверки.

Третий пример --- <<Оформление студенческих работ. 4 шага!>> --- предполагает базовую проверку и исправления, но требует предварительной подготовки: документ необходимо загружать без оглавления, а заголовки и подзаголовки вводить вручную в специальные поля \cite{serv3}. Интерфейс сервиса выглядит устаревшим и неудобным, а представленные на сайте данные, такие как количество проверок в день и оформленных работ, вызывают сомнения в их достоверности (рис. \ref{fig:pic113}). Система выше использует устаревший ГОСТ, имеет сложную ручную подготовку к проверке, не инициализирует проверку работы, не дает возможности настраивать правила проверки.

\myfigure{0.8}{serv31}{Скриншот сервиса <<Оформление студенческих работ. 4 шага!>>}{pic113}



Существуют системы, которые предоставляют инструмент для автоматизации отдельного элемента оформления, такого как список использованных источников, которые работают достаточно качественно.

% SWOT-анализ разрабатываемой системы

Анализ аналогичных решений показал, что на текущий момент не существует системы, которая полноценно соответствовала бы требованиям студентов и нормоконтролеров Иркутского государственного университета. На основании изученных аналогичных систем и выявленных потребностей пользователей был проведен SWOT-анализ разрабатываемой системы автоматического нормоконтроля.

SWOT-анализ показывает, что разрабатываемая система обладает значительными преимуществами перед существующими решениями, учитывает ключевые потребности пользователей и имеет потенциал для дальнейшего развития (рис. \ref{fig:pic114}). Однако для успешной реализации необходимо провести тщательное тестирование алгоритмов проверки для успешной реализации.

\myfigure{0.8}{swot2}{SWOT-анализ разрабатываемой системы автоматического нормоконтроля}{pic114}

%\section{Анализ нормативной базы}


\section{Проектирование информационной системы}

Проектирование информационной системы является ключевым этапом, направленным на создание эффективного решения для автоматизации нормоконтроля студенческих работ. Этот процесс включает в себя определение назначения, требований, функций, алгоритмов и архитектуры системы, а также разработку диаграмм. Большинство диаграмм разрабатываются в нотации UML (Unified Modeling Language), которая является универсальным языком визуализации, спецификации и документирования сложных систем. Использование UML обеспечивает единый стандарт для описания всех аспектов системы, что упрощает её проектирование и реализацию. В результате проектирования формируется основа для реализации системы, которая будет удовлетворять потребности всех участников процесса и обеспечивать удобное использование.

Перед непосредственным проектированием была разработана концепция системы, определяющая её основные цели, продукты проекта, функции и результаты. В рамках концепции были проанализированы ключевые участники, допущения и ограничения, ресурсы проекта, риски, посчитаны сроки проекта. Были сформулированы критерии приемки и обоснование полезности проекта.

Назначение информационной системы

Информационная система автоматического нормоконтроля предназначена для автоматизации проверки оформления выпускных квалификационных работ, курсовых работ и отчетов по практике в соответствии с установленными стандартами и требованиями. 
Основными целями системы являются:
\begin{enummarker}
\item ускорение процесса проверки;
\item снижение нагрузки на нормоконтролеров;
\item повышение точности проверки;
\item повышение оперативности выдачи результатов проверки.
\end{enummarker}

Система должна позволить существенно сократить время, затрачиваемое на проверку документов, за счет автоматизации рутинных задач. Она берет на себя выполнение операций, таких как проверка соответствия ГОСТ и требованиям университета, что освобождает нормоконтролеров для выполнения более сложных задач. Исключение человеческого фактора снижает вероятность пропуска ошибок или неправильного их выявления, обеспечивая строгое соответствие стандартам. Обеспечение быстрого доступа к результатам проверки позволяет пользователям оперативно вносить изменения в свои документы.


% Основными пользователями системы являются: нормоконтролер --- специалист, осуществляющий контроль за оформлением документов, \break студент --- пользователь, загружающий свои работы для проверки.

Диаграмма бизнес-процесса позволяет визуализировать и понять работу системы автоматического нормоконтроля в общих чертах (рис. \ref{fig:pic11}).

\myfigure{0.99}{business-process}{Диаграмма бизнес-процесса <<Нормоконтроль студенческой работы>>}{pic11}

В данном процессе возможно участие двух основных акторов: Студент и Нормоконтролер. Они связаны с событием <<Начат процесс нормоконтроля>>, которое инициирует проверку документа. Далее событие направляется в бизнес-процесс <<Нормоконтроль студенческой работы>>. На выходе этого процесса формируется проверенная студенческая работа с найденными ошибками и соответствующими рекомендациями по исправлению. Бизнес-процесс задействует ресурсы: сервер и сайт, которые обеспечивают техническую реализацию системы. Также процесс использует информационные ресурсы, такие как ГОСТ и требования университета, для проверки документов. Целью бизнес-процесса является обеспечение быстрого и правильного нормоконтроля оформления студенческой работы.



Требования к информационной системе

Информационная система автоматического нормоконтроля включает в себя функциональные, нефункциональные, технические и другие требования, которые обеспечивают её эффективность, производительность и удобство использования.

Функциональные требования содержатся на диаграмме \break требований
(рис. \ref{fig:pic12}). Эта диаграмма отражает ключевые аспекты работы системы и охватывает три основных направления: загрузка документов, проверка документа и выдача результатов проверки. Каждое направление агрегируется на более подробные и точные требования.

\myfigure{0.99}{reqs}{Диаграмма требований к системе автоматического нормоконтроля}{pic12}





К нефункциональным требованиям системы относятся:
\begin{enummarker}
\item обработка одного документа за время, не превышающее трёх минут;
\item одновременная работа как минимум 20 пользователей;
\item корректная работа на различных браузерах и устройствах.
\end{enummarker}

\break

Технические требования включают:
\begin{enummarker}
\item использование серверного решения для обработки и хранения данных; %облачного
\item логирование действий пользователей и событий для мониторинга; %и диагностики
\item интуитивно понятный и простой интерфейс.
\end{enummarker}
%%% можно еще дописать требования из ТЗ (требования к видам обеспечния(лингвистическое методическое программное информационное))






Более того, была разработана диаграмма интерфейсов системы (рис. \ref{fig:pic13}). 

\myfigure{0.8}{User-Interface}{Диаграмма пользовательских интерфейсов системы автоматического нормоконтроля}{pic13}

Диаграмма включает три ключевых экрана, обеспечивающих взаимодействие пользователей с системой. Экран загрузки документа предоставляет возможность выбора типа работы, загрузки файла в установленных форматах и инициализации проверки с помощью кнопки. Экран процесса проверки визуализирует текущий этап анализа документа, предоставляя пользователю обратную связь о ходе проверки. Экран результатов проверки представляет детализированный отчет об ошибках и рекомендации по их исправлению, обеспечивая пользователю доступность и наглядность информации для внесения необходимых корректировок. Эти интерфейсы разработаны с учетом требований к простоте и интуитивной понятности.

%Предложенные функциональные, нефункциональные, технические требования и интерфейсы формируют основу для разработки системы.% нормоконтроля. 
%Особое внимание уделено автоматизации рутинных задач для достижения целей системы.

Функции информационной системы

%%% Диаграмма Use Case
Информационная система автоматического нормоконтроля реализует функции, обеспечивающие автоматизацию проверки оформления студенческих работ. Она позволяет загружать документы, проверять их на соответствие установленным стандартам и требованиям, а также предоставлять отчёт с выявленными ошибками и рекомендациями по их исправлению.

Диаграмма вариантов использования (Use Case) --- это инструмент моделирования, который позволяет визуализировать взаимодействие пользователей с функциональностью системы. Она отображает основные действия, выполняемые в системе, их взаимосвязи, а также роли \break пользователей (рис. \ref{fig:pic14}). 

\myfigure{1}{UseCases_verysmall}{Диаграмма вариантов использования системы автоматического нормоконтроля}{pic14}

На диаграмме представлены основные функции системы автоматического нормоконтроля и их взаимодействие с пользователями. Акторы \break системы --- это Cтудент и Нормоконтролер, объединенные в актора Пользователь. Актор Пользователь взаимодействует с функцией <<Проверка оформления работы>>, которая включает в себя последовательные \break этапы: загрузку файла работы и получение отчета о правильности оформления. При загрузке файла работы пользователь выбирает нужный файл и указывает тип документа, что позволяет системе учитывать специфику проверки. По результатам проверки работы пользователь может ознакомиться с процентной оценкой соответствия требованиям, обнаруженными ошибками и рекомендациями по их исправлению. 

Актор Нормоконтролер, в свою очередь, имеет доступ к функции редактирования правил проверки, что позволяет адаптировать критерии анализа в соответствии с актуальными стандартами. 

Диаграмма демонстрирует, как система организует взаимодействие между пользователями и ее функциональностью, обеспечивая автоматизацию проверки и управление правилами проверки.



Основные алгоритмы информационной системы

%%% Диаграмма деятельности диаграмма последовательности
Алгоритмы информационной системы автоматического нормоконтроля являются основой её работы, обеспечивая выполнение задач проверки оформления документов. Они описывают последовательность шагов, необходимых для обработки данных, анализа их соответствия требованиям и выдачи результатов пользователю. В данной системе алгоритмы охватывают процессы загрузки документа, выполнения проверки, анализа ошибок, формирования отчета и предоставления результатов пользователю. Для наглядного представления этих процессов используются диаграммы деятельности (activity model) и последовательности (sequence model), которые позволяют визуализировать взаимодействие между элементами системы и пользователями, а также порядок выполнения операций.

Диаграмма деятельности представляет собой графическое отображение последовательности действий и решений в процессе взаимодействия пользователя с системой (рис.  \ref{fig:pic15}). Она организована в три дорожки: Пользователь, Клиент и Сервер, что позволяет разделить действия между фронтендом, бэкендом и пользователем. 

Процесс начинается с того, что пользователь открывает сайт, где клиент отображает форму для загрузки документа. Пользователь выбирает файл, указывает тип работы и нажимает кнопку <<Начать проверку>>. Далее происходит проверка корректности загруженного файла. В случае некорректности файла пользователь получает уведомление об ошибке, и процесс завершается. Если файл корректен, клиент отображает экран <<Идет процесс проверки...>>, и система переходит к анализу документа на сервере.

\myfigure{1}{Activity-model}{Диаграмма деятельности проверки студенческой работы в системе}{pic15}

   Основной этап проверки происходит на сервере: анализ соответствия документа установленным правилам. Если обнаружены несоответствия, фиксируются ошибки, рассчитывается процент соответствия и подбираются рекомендации. Если ошибок нет, процент соответствия автоматически составляет 100\%. После этого формируется отчет о результатах проверки, который передается клиенту для отображения пользователю. Пользователь может просмотреть отчет, при необходимости исправить ошибки и инициировать повторную проверку, что замыкает процесс на этапе загрузки файла.

Диаграмма последовательности отображает взаимодействие между пользователем, интерфейсами (граничными объектами) системы, контроллерами и сущностями данных в процессе проверки (рис.  \ref{fig:pic16}). 

\myfigure{0.999}{Seq1small}{Диаграмма последовательности проверки оформления работы}{pic16}

Пользователь начинает взаимодействие, заполняя поля формы и нажимая кнопку <<Начать проверку>> в boundary-объекте <<Форма загрузки студенческой работы>>. Эта форма инициирует выбор правил проверки, передавая запрос в контроллер <<Выбор нужного набора правил на основе типа работы>>. Контроллер обращается к сущности <<Правила проверки>>, откуда извлекаются соответствующие требованиям правила. Затем форма запускает процесс обработки файла через контроллер <<Обработка файла>>, который создает сущность <<Документ>>. Контроллер также выполняет проверку на соответствие правил, анализируя документ, и формирует сущность <<Результаты проверки>>. В сущности фиксируются найденные несоответствия, процент выполнения требований и рекомендации. 



Дополнительно разработана диаграмма последовательности, отображающая процесс обновления правил проверки (рис.  \ref{fig:pic17}). Актор <<Нормоконтролер>> инициирует изменения значений необходимых правил через граничный объект <<Файл с правилами проверки>>. После внесения корректировок данные сохраняются и обрабатываются контроллером <<Обновление правил>>. Контроллер передает команду <<обновитьПравилаПроверки()>> сущности <<Правила проверки>>, где изменения фиксируются и сохраняются в системе. Эта диаграмма демонстрирует логику взаимодействия нормоконтролера с системой для поддержания актуальности правил проверки, что позволяет адаптировать алгоритмы системы к новым требованиям ГОСТ или университета.

\myfigure{0.9}{Seq2small}{Диаграмма последовательности <<Редактирование правил проверки>>}{pic17}

Диаграммы деятельности и последовательности дают полное представление о логике работы системы автоматического нормоконтроля. В совокупности эти модели обеспечивают упорядоченный подход к проектированию алгоритмов системы, делая её процессы прозрачными.

Диаграмма классов информационной системы

Диаграмма классов является одним из ключевых инструментов проектирования, отображающим структуру информационной системы через классы, их атрибуты, методы и взаимосвязи. Она описывает статическую архитектуру системы %Для системы автоматического нормоконтроля диаграмма классов имеет особую важность, поскольку напрямую определяет, каким образом будет разрабатываться система. 
и напрямую определяет, каким образом будет разрабатываться система. 
Диаграмма классов информационной системы отражает ключевые элементы и их взаимосвязи, сформированные в соответствии с функциональной структурой системы (рис. \ref{fig:pic18}). 

Класс <<Документ>> описывает студенческую работу, включая такие атрибуты, как содержание (content), тип документа и результаты проверки. Для проверки документов используется класс <<Правила проверки>>, хранящий набор правил проверки для различных типов документов, таких как дипломные работы, курсовые или отчеты по практике, что задается перечислением <<Тип документа>>. Правила представлены классом <<Правило>> и включают свойство, условие и тип правила, заданный перечислением <<Тип правила>>. Класс <<Результат проверки>> объединяет результаты проверки, включая списки ошибок и рекомендации. Каждый объект ошибки связан с правилом, вызвавшим её, а рекомендации предоставляют способы исправления.

\myfigure{0.98}{ClassModel_small (1)}{Диаграмма классов системы автоматического нормоконтроля}{pic18}



Для обработки загрузки и проверки документов система включает класс boundary-стереотипа <<Форма загрузки документа>>, и классы control-стереотипа <<Обработка файла>>, <<Выбор правил>>, <<Поиск несоответствий в оформлении>>. Класс <<Форма загрузки документа>> отвечает за взаимодействие с пользователем при загрузке документа. <<Обработка файла>> обрабатывает загруженный файл и преобразует его в объект Document. <<Выбор правил>> выбирает набор правил проверки, подходящий для типа документа. Класс <<Поиск несоответствий в оформлении>> выполняет проверку документа на соответствие выбранным правилам и формирует объект <<Результат проверки>>. 


В результате диаграмма классов демонстрирует такой подход к проектированию системы, который позволяет легко её адаптировать и масштабировать. Например, добавление новых типов документов или правил проверки может быть реализовано без существенных изменений существующей архитектуры. Также диаграмма раскрывает важные взаимосвязи между компонентами системы, которые обеспечивают их согласованное взаимодействие.

Архитектура информационной системы

Архитектура информационной системы предоставляет целостное представление о структуре, взаимодействиях и взаимосвязях между различными частями системы. Спроектированная архитектура позволяет разработчикам, аналитикам и заказчикам формировать единое видение системы, что упрощает её реализацию и дальнейшую поддержку.

Диаграмма компонентов отображает разбиение системы на логические части (компоненты), их функциональную направленность и взаимодействия между ними. На диаграмме компонентов системы автоматического нормоконтроля показано, как модули системы коммуницируют между \break собой (рис. \ref{fig:pic19}). 

\myfigure{1}{Components_verysmall}{Диаграмма компонентов системы автоматического нормоконтроля}{pic19}

Основной компонент <<Слой API main.py>> отвечает за обработку пользовательских запросов и делегирует выполнение задач специализированным модулям. Он связан связью <<delegate>> с компонентами <<Модуль проверки>>, <<Модуль управления документами>> и <<Модуль управления правилами>>. Эти модули выполняют основную бизнес-логику: проверяют документы, управляют их данными и предоставляют правила для проверки. Все перечисленные модули, включая <<Слой API main.py>>, используют компонент <<Модуль логгирования logger.py>>, который обеспечивает регистрацию действий системы, упрощая отладку и мониторинг.

Диаграмма развертывания UML показывает, как программные компоненты размещаются на физических или виртуальных узлах (нодах), а также их взаимодействия. Этот тип диаграммы позволяет понять, где и как выполняются элементы системы, какие устройства или серверы задействованы, и каким образом обеспечивается коммуникация между ними.

На диаграмме развертывания системы автоматического нормоконтроля представлены две основные ноды: <<Персональный компьютер пользователя>> и <<Сервер>> (рис.  \ref{fig:pic110}). Персональный компьютер пользователя включает компонент <<Браузер>>, который предоставляет интерфейс системы. Сервер содержит модули системы, обеспечивающие выполнение бизнес-логики. Между компонентом <<Браузер>> на стороне клиента и компонентом <<Слой API main.py>> на сервере реализована связь <<request/response>>, обеспечивающая обмен данными и обработку пользовательских запросов. 

\myfigure{1}{Deployment1_verysmall}{Диаграмма развертывания системы автоматического нормоконтроля}{pic110}

Такая архитектура демонстрирует распределённую природу системы, где клиентская часть отвечает за взаимодействие с пользователем, а серверная --- за выполнение операций.

\section{Обзор инструментов, используемых в разработке системы}

Для реализации системы автоматизации нормоконтроля студенческих работ выбран набор технологий с учетом функциональных требований проекта. Язык программирования Python стал очевидным выбором, поскольку библиотека dedoc, хорошо подходящая для обработки текстовых документов, написана на этом языке. Библиотека dedoc была выбрана после анализа доступных решений, так как она наиболее полно соответствует потребностям обработки различных форматов документов и обеспечивает высокую точность извлечения текстовой информации \cite{dedoc}. Для обработки DOCX-документов дополнительно используется библиотека python-docx, которая позволяет открывать, анализировать и извлекать структурированное содержимое из файлов формата Microsoft Word, что необходимо для более детальной проверки оформления. Еще одним немало важным инструментом данной системы выступает библиотека re, являющаяся частью стандартной библиотеки Python, активно применяется как в модуле обработки DOCX, так и при работе с LaTeX-файлами — она обеспечивает гибкий механизм поиска и извлечения шаблонов текста с помощью регулярных выражений, что необходимо при проверке на соответствие форматированию и структуре.

FastAPI используется для создания серверного API, обеспечивающего взаимодействие между клиентской и серверной частями системы \cite{fastapi}. Git служит инструментом контроля версий, позволяющим удобно управлять изменениями в проекте, работать с ветками и отслеживать историю разработки. Инструмент тестирования unittest используется для модульного тестирования компонентов системы с целью выявления и предотвращения ошибок на раннем этапе. Для реализации Telegram-бота, предоставляющего способ взаимодействия с системой, применяется асинхронная библиотека aiogram, которая обеспечивает стабильную и масштабируемую работу с API Telegram и позволяет обрабатывать команды пользователей, маршрутизируя обращения к соответствующим функциям backend-сервиса.

Кроме того, в различных модулях системы используются дополнительные библиотеки стандартной и внешней экосистемы Python для реализации необходимой функциональности, такие как argparse для создания CLI-интерфейса, datetime и json для работы с временными метками и данными, typing для поддержки аннотаций типов, os и tempfile для работы с файловой системой, sqlite3 для взаимодействия с локальной базой данных, requests для выполнения HTTP-запросов, а также logging для ведения журнала событий и отладки.

\section*{Выводы по главе}
\addcontentsline{toc}{section}{Выводы по главе}

В ходе работы был проведён анализ аналогичных систем и выявлены потребности пользователей, которые показали необходимость разработки специализированного инструмента для автоматической проверки студенческих работ, учитывающего требования ГОСТ и Иркутского государственного университета. Проектирование информационной системы включало разработку диаграмм, таких как диаграмма бизнес-процесса, требований, интерфейсов, вариантов использования, деятельности, последовательности, классов, компонентов и развёртывания, обеспечивающих структурированное представление системы и её функциональных возможностей. В ходе обзора инструментов для разработки были выбраны технологии, соответствующие требованиям системы, включая Python, FastAPI, dedoc, python-docx, aiogram, Git и unittest.


\chapter{РАЗРАБОТКА СИСТЕМЫ АВТОМАТИЧЕСКОГО НОРМОКОНТРОЛЯ}
% снова обнуляем счетчики section, subsection и equation в новой главе
\setcounter{section}{0}
\setcounter{subsection}{0}
\setcounter{equation}{0}
\section{Файловая структура системы}

В процессе разработки системы автоматического нормоконтроля структура проекта значительно разрослась, что обусловлено сложностью логики обработки документов и необходимостью поддержки различных форматов (рис. \ref{fig:pic_project_structure}). Изначально при проектировании системы я стремилась к созданию чистой архитектуры с использованием шаблонов проектирования и принципов SOLID. Однако, на практике оказалось, что сочетание написания рабочего кода и строгого следования принципам проектирования требует значительно больше времени. Поэтому текущая структура проекта ещё не является окончательной и будет изменяться и улучшаться по мере его развития.

\myfigure{0.95}{project_structure}{Структура python-проекта <<Системы автоматического нормоконтроля>>}{pic_project_structure}

На данный момент проект состоит из нескольких основных модулей. 
Папка docs содержит тестовые документы, используемые для проверки работы системы. В ней хранятся файлы в форматах DOCX, PDF и LaTeX. 
Каталог rules включает JSON-файлы с правилами нормоконтроля, предназначенными для различных типов работ, таких как дипломные проекты, курсовые работы и отчёты по практике. 

Основной код проекта расположен в каталоге src, который организован в несколько подпапок. 
Раздел core включает в себя основные типы и абстракции, такие как базовые классы для логики и моделей (abstract\_logic.py, abstract\_model.py), вспомогательные enum-перечисления (doc\_type.py, event\_type.py, logging\_level.py, rule\_type.py), а также модуль validator.py, который содержит классы пользовательских исключений и выполняет проверку аргументов на соответствие требованиям по типу и длине. 
Раздел logics содержит основную логику обработки и проверки документов. Подраздел checkers включает модули для проверки документов различных форматов: base\_checker.py (базовый класс для всех проверок), а также специализированные модули docx\_checker.py, latex\_checker.py и pdf\_checker.py, отвечающие за проверку соответствующих форматов. 
В подразделе parsers представлены модули для разборки документов различных типов: docx\_parser.py (разбор DOCX-документов), latex\_parser.py (разбор LaTeX-документов) и pdf\_parser.py (разбор PDF-документов). 
Также в папке logics находятся модули doc\_service.py и rule\_service.py, которые помогают в управлении документами и правилами. Файл logging.py содержит код, необходимый для логирования работы системы, а observe\_service.py отвечает за отслеживание событий в системе и выполнение определённых действий при их возникновении. Следует отметить, что pdf\_parser.py и pdf\_checker.py ещё не были реализованы.

Раздел models содержит модели данных, включая document.py (описание структуры документа), mistake.py (модель ошибки в оформлении), recommendation.py (модель рекомендаций по исправлению), а также rule.py, validation\_rules.py и validation\_result.py, описывающие правила и результаты нормоконтроля. Дополнительно в этом разделе представлен модуль settings\_model.py, содержащий модель настроек системы. Однако, из-за сложности применения, модели данных пока не используются в коде, но в дальнейшем планируется их использование в системе.

Папка tests содержит файлы с юнит-тестами для проверки работы различных элементов системы.

К конфигурационным файлам относятся .gitignore (исключение ненужных файлов из репозитория), requirements.txt (список зависимостей проекта) и settings.json (конфигурационный файл настроек системы).

Кроме того, в проекте присутствует файл main.py, в котором определены маршруты и функции FastAPI, обеспечивающие взаимодействие с системой.

Файлы example\_lib.py и example\_req.py содержат примеры работы с библиотекой dedoc.

Таким образом, структура проекта охватывает все необходимые компоненты для автоматического нормоконтроля документов. В дальнейшем планируется оптимизация кода, рефакторинг и совершенствование архитектуры.

% \section{Разработка моделей данных}

% Модели данных являются важной частью информационной системы, так как они обеспечивают структурированное представление информации, используемой в процессе нормоконтроля. Использование моделей данных упрощает валидацию, хранение и передачу данных между различными частями системы. Они делают код более читаемым и позволяют избежать дублирования логики.

% Разработка моделей данных в данной системе основывалась на предварительно созданной диаграмме классов, разработанной в рамках проектирования системы. Диаграмма классов позволила определить ключевые сущности, их атрибуты и взаимосвязи, что обеспечило согласованную и логичную структуру моделей.

% Все модели данных обладают рядом общих характеристик. Они наследуются от базового класса AbstractModel, что обеспечивает единообразие структуры. Модели имеют методы для сериализации, десериализации и сравнения объектов. Внутри моделей используется механизм валидации данных с помощью класса Validator, что гарантирует корректность передаваемых значений. 

% Каждая из разработанных моделей имеет свою специфику. Модель Document представляет собой сущность документа, содержащую информацию о его типе (DocType) и результатах проверки (ValidationResult). Модель Mistake предназначена для хранения информации об ошибках оформления, выявленных в документе. Она включает в себя ссылку на нарушенное правило (Rule) и текстовое описание ошибки. Модель Recommendation используется для формирования рекомендаций по исправлению ошибок. Она содержит ссылку на соответствующую ошибку (Mistake) и текст рекомендации. Модель Rule описывает правила нормоконтроля, включающие проверяемый атрибут, тип правила (RuleType) и условие проверки. Модель ValidationResult агрегирует информацию о результатах проверки, включая список выявленных ошибок и соответствующих рекомендаций. Модель ValidationRules определяет набор правил нормоконтроля, применяемых к конкретному типу документа. Также в системе присутствует модель SettingsModel, предназначенная для хранения настроек системы, таких как уровень логирования (LoggingLevel).

% Кроме базового класса для моделей был разработан базовый класс AbstractLogic, который обрабатывает логику, связанную с событиями и ошибками. Ещё для удобства и дополняемости системы были созданы enum-классы DocType, EventType, LoggingLevel и RuleType, которые определяют допустимые типы документов, событий, уровней логирования и типов правил соответственно.

% На текущем этапе разработки модели данных не используются в системе, поскольку пока нет окончательного представления о том, как именно они должны быть интегрированы в процесс нормоконтроля. Дальнейшая работа над проектом предполагает доработку архитектуры системы для полноценного использования моделей данных.

\section{Разработка модуля управления правилами проверки}

Модуль управления правилами проверки предназначен для доступа к правилам нормоконтроля и их редактирования. Эти правила хранятся в JSON-файлах, например, diploma\_rules.json, который содержит требования к выпускной квалификационной работе (рис. \ref{fig:diploma_rules_big}). 

\myfigure{0.9}{diploma_rules_big}{Отрывок json-файла с правилами проверки ВКР}{diploma_rules_big}
% переделать сделать составную картинку

В этом файле определены общие требования к оформлению документа, такие как шрифт, размер шрифта, межстрочный интервал, параметры выравнивания и отступы. Также задаются правила структуры, включая обязательные главы, разделы и ключевые слова, необходимые во введении. Дополнительно описаны требования к оформлению заголовков глав, разделов и других элементов документа.

Для курсовых работ и отчетов по практике также предусмотрены аналогичные JSON-файлы с правилами: course\_work\_rules.json и practice\_report\_rules.json. Они содержат специфические требования для каждого типа документа, что позволяет гибко адаптировать систему под разные форматы учебных работ.

Обработка этих файлов выполняется в классе RuleService. Он предоставляет методы для загрузки, сохранения и обновления правил, используя механизм работы с JSON. Метод load\_rules загружает правила из соответствующего файла на основе типа документа. Метод save\_rules позволяет сохранять изменения в файле, а update\_rule предоставляет возможность точечного изменения отдельных параметров правил. Кроме того, класс RuleService поддерживает получение списка всех доступных типов правил.

Работа RuleService интегрирована в API системы. Это позволяет пользователям запрашивать и изменять правила без непосредственного взаимодействия с файловой системой. Подробное описание API и его возможностей будет описано в дальнейшем.

\section{Разработка проверки структуры студенческой работы}

Проверка структуры студенческой работы включает в себя контроль наличия обязательных глав и разделов, а также проверку введения на наличие ключевых слов. В работе должны присутствовать такие главы, как введение, первая и вторая главы с разбивкой на разделы, заключение, список использованных источников и приложения. Титульный лист и содержание тоже обязательны в структуре любой работы. Кроме того, во введении должны быть ключевые слова: цель, задачи, актуальность, объект исследования, предмет исследования, теоретическая новизна и практическая значимость.

Первым был реализован механизм проверки структуры документов в формате LaTeX. В данной системе проверка структуры и оформления LaTeX-документа осуществляется по шаблону научно-исследовательской работы, который был разработан мной в рамках курсовой работы в прошлом году. Этот шаблон задает базовые требования к структуре и форматированию работы, что позволяет использовать его в качестве эталона при автоматической проверке.

На этапе разработки рассматривалась возможность использования библиотеки TexSoup для парсинга LaTeX-файлов. Однако оказалось, что TexSoup не полностью подходит для решения данной задачи. В связи с этим было принято решение реализовать собственный механизм парсинга с использованием регулярных выражений.

Для обработки LaTeX-документов был создан класс LatexParser. Его основная задача —-- извлекать структуру документа, а также анализировать введение. При инициализации LatexParser получает содержимое .tex-файла (без учета закомментированных строк с помощью функции remove\_comments), запускает метод run\_parse(), который выделяет ключевые структурные элементы, а затем выполняет дополнительные проверки через run\_checks(). Метод parse\_structure() ищет все нумерованные и ненумерованные главы и разделы, формируя удобное представление структуры документа. parse\_title\_and\_toc() проверяет наличие титульного листа и содержания, а parse\_addcontentsline() контролирует корректность использования команды для отображения ненумерованных глав в содержании.

Проверка структуры документа осуществляется в классе LatexChecker. Этот класс использует LatexParser для получения структуры документа и сравнивает её с требованиями по структуре, загружаемыми из JSON-файла нужного типа работы. Метод check\_structure() сверяет наличие обязательных глав и разделов, а check\_introduction\_keywords() проверяет введение на наличие ключевых слов. Если какой-либо обязательный элемент отсутствует, в список ошибок добавляется соответствующее сообщение (рис. \ref{fig:structure_errors}).

\myfigure{0.9}{structure_errors}{Пример списка ошибок по результатам проверки структуры документа}{structure_errors}

Кроме того, была разработана система проверки структуры документов в формате DOCX. Этот процесс основан на анализе содержимого документа и его структуры, что позволяет выявлять наличие или отсутствие обязательных разделов, а также находить повторяющиеся элементы.

Для парсинга документов формата DOCX был разработан класс DocxParser. Он использует библиотеку Dedoc для извлечения структуры документа. При инициализации DocxParser принимает путь к файлу и вызывает метод run\_parse(), который управляет процессом парсинга. Внутри этого метода создается объект DedocManager(), представляющий собой основной интерфейс взаимодействия с библиотекой Dedoc. Далее вызывается метод parse(), который принимает путь к документу и тип документа. В данном случае передается параметр, обозначающий студенческую работу, что позволяет Dedoc применять наиболее подходящие алгоритмы обработки. Результатом работы parse() является объект, содержащий структурированное представление документа, которое затем сериализуется. Метод parse\_structure() извлекает список всех найденных в документе глав. Эти данные используются для дальнейшей проверки документа.

Для автоматического контроля структуры работы разработан класс DocxChecker. Он получает данные от DocxParser и сравнивает их с требованиями, загружаемыми из JSON-файла. Метод check\_structure() проверяет наличие всех обязательных глав, сравнивая их с найденными в документе. Если какой-либо элемент отсутствует, он добавляется в список ошибок. Кроме того, реализована проверка на повторяющиеся разделы, что позволяет обнаруживать дублирование заголовков.

Итак, реализованные модули позволяют автоматически анализировать структуру работ формата LaTeX и DOCX, выявлять несоответствия требованиям. В дальнейшем проверка структуры будет совершенствоваться, охватывая ещё больше важных аспектов, а также предлагать рекомендации по исправлению ошибок.

%%%% + работа библиотеки dedoc для парсинга docx в DocxParser %%%%

\section{Разработка проверки правил оформления студенческой работы}

Проверка правил оформления студенческой работы включает в себя контроль соответствия документа установленным требованиям по форматированию. Это касается размера шрифта, межстрочных интервалов, отступов, оформления заголовков, подписей к рисункам и таблицам, оформления списка использованных источников и других элементов работы.

В системе проверка правил оформления осуществляется через класс LatexChecker. В этом классе находится функция check\_sty\_file(), которая отвечает за проверку корректности используемого .sty-файла. В LaTeX-шаблоне научно-исследовательской работы .sty-файл содержит настройки форматирования документа. В данной системе загружен эталонный файл settings.sty, который используется в качестве образца, с которым сравнивается загруженный пользователем .sty-файл.

Функция check\_sty\_file() начинает свою работу с нахождения эталонного файла settings.sty. Далее загружается .sty-файл, прикрепленный пользователем, и его содержимое построчно сравнивается с эталонным. Для корректного сравнения из файлов удаляются комментарии и пустые строки. Если на каком-либо этапе обнаруживаются несовпадения в строках, формируется сообщение об ошибке с указанием строки, где найдено расхождение. Также проверяется общее количество строк в файлах: если загруженный .sty-файл содержит больше или меньше строк, чем эталонный, это также фиксируется как ошибка. Использование .sty-файла в качестве основного источника правил оформления позволяет стандартизировать процесс проверки.

Дополнительно необходимо рассмотреть вариант проверки строк .sty-файла в случае изменения их порядка. Хотя порядок строк в данном файле, как правило, должен оставаться неизменным, возможны ситуации, когда пользователи переставляют команды местами, не нарушая логики форматирования. В таком случае текущий алгоритм проверки будет отмечать это как ошибку, поэтому стоит продумать механизм, который позволит учитывать такие изменения без потери корректности проверки.

В системе также реализована проверка размера шрифта и жирности заголовков глав для документов формата DOCX. Эта проверка выполняется с помощью классов DocxParser и DocxChecker, в которых анализируются аннотации форматирования, извлеченные с помощью библиотеки Dedoc.

На этапе парсинга документа DocxParser вызывает метод run\_parse(), который загружает документ в DedocManager и получает его структурированное представление. Затем метод parse\_font\_details() проходит по найденным заголовкам глав и разделов, извлекая информацию о размере шрифта и его жирности. Данные хранятся в виде словаря, где ключами являются заголовки, а значениями — их форматирование (размер шрифта и наличие жирного начертания).

Для выполнения проверки в DocxChecker загружаются правила оформления, содержащие ожидаемый размер шрифта и стиль заголовков. Затем метод check\_font\_size() проходит по заголовкам и сравнивает их параметры с установленными требованиями. Если размер шрифта заголовка не совпадает с ожидаемым, в список ошибок добавляется соответствующее сообщение.

Однако на практике возникла проблема с проверкой жирности заголовков. Если заголовок главы оформлен без жирного начертания, Dedoc перестает воспринимать его как заголовок и не включает его в структуру документа. В результате такие заголовки не проходят проверку на соответствие требованиям, но и не фиксируются в системе как ошибки. Это ограничение библиотеки Dedoc, которое влияет на точность проверки, и в будущем потребуется разработка обходного решения для учета таких случаев.

В ближайшее время будет проводиться рефакторинг проверяющих модулей. Это необходимо для улучшения их архитектуры, повышения читаемости кода и упрощения добавления новой логики проверки.

Подводя итоги по проверке оформления, LaTeX проверяется с помощью сравнения стилевого файла пользователя с эталонным стилевым  файлом, а реализованный механизм для DOCX позволяет выявлять ошибки в размере шрифта глав работы.


\section{Разработка API системы автоматического нормоконтроля}

API (Application Programming Interface, или программный интерфейс приложения) представляет собой набор правил и инструментов, которые позволяют различным программным системам взаимодействовать между собой. API позволяет интегрировать систему с различными платформами, что делает её более гибкой и универсальной.

В данном проекте API предоставляет набор маршрутов (эндпоинтов), которые позволяют пользователю или внешнему сервису взаимодействовать с системой нормоконтроля (рис. \ref{fig:api1}). В качестве внешнего сервиса может выступать сайт, мобильное приложение или Telegram-бот.

\myfigure{0.999}{api1}{Скриншот интерактивной документации по API}{api1}


Рассмотрим каждый реализованный на данный момент запрос в API системы более подробно. Запрос /api/documents/options возвращает список доступных типов документов, которые система может проверять. Этот запрос нужен для того, чтобы предложить пользователю выбор типа документа для проверки, такого как диплом, курсовая работа или отчет по практике. При обработке запроса происходит вызов метода get\_doc\_types из сервиса DocService, который извлекает данные о типах документов и передает их в ответ.

Запрос /api/rules/options аналогично возвращает список доступных типов правил для проверки документов. Данные для этого запроса собираются с помощью метода get\_rule\_types из сервиса RuleService. Скорее всего, данный запрос не останется в итоговом варианте системы из-за отсутствия необходимости пользователю видеть типы правил.

Запрос /api/rules/{doc\_type} предоставляет конкретные правила для проверки документа определенного типа. В запрос передаётся параметр doc\_type, который указывает на тип документа, например, диплом или курсовая работа. Система на основе этого параметра извлекает соответствующие правила через метод load\_rules в сервисе RuleService. Если тип документа неверен или не поддерживается, система возвращает ошибку с кодом 400.

Запрос /api/rules/update позволяет обновить правило для конкретного типа документа. В теле запроса передаются параметры doc\_type, rule\_key и new\_value, которые указывают на тип документа, ключ обновляемого правила и новое значение для этого правила. При обработке запроса используется метод update\_rule из сервиса RuleService, который обновляет данные в системе и возвращает сообщение о результатах.

Запрос /api/rules/update/all аналогичен предыдущему, но обновляет правило для всех типов документов. Этот запрос полезен, когда необходимо внести изменения в правила для всех поддерживаемых типов документов сразу. Обновление производится по аналогии с предыдущим запросом, но изменения распространяются на все типы документов.

Запросы /api/documents/validate/single\_file и /api/documents/validate/latex выполняют проверку документов на соответствие заданным правилам. Первый запрос принимает файл в формате DOCX и тип документа, а \break второй --— два файла LaTeX (файл .tex и файл стилей .sty). В обоих случаях система проверяет, соответствует ли документ требованиям по структуре и оформлению. Для этого используются соответствующие проверяющие \break файлы —-- DocxChecker для файлов DOCX и LatexChecker для файлов LaTeX. Проверка осуществляется с использованием методов этих классов, которые анализируют документ и возвращают результаты проверки. В случае ошибок загрузки файлов или несоответствия форматов API возвращает ошибку с соответствующим сообщением.

В дальнейшем, при необходимости, API приобретёт дополнительные запросы.

\section{Разработка логирования в системе}

% добавить инфу про settings manager

Логирование является важной частью любой информационной системы, позволяя отслеживать её работу, диагностировать ошибки и анализировать события, происходящие в процессе функционирования. Система логирования записывает информацию о событиях в файлы или другие хранилища, что позволяет разработчикам и администраторам быстро находить проблемы и анализировать производительность системы.

В системе автоматизированного нормоконтроля логирование организовано с помощью класса Logging, который управляет записью событий в лог-файл. Основной класс Logging реализует обработку событий через механизм наблюдателя (ObserveService). Логирование основано на нескольких уровнях, определяемых перечислением LoggingLevel, включающим INFO, ERROR и DEBUG. Запись сообщений в лог происходит в зависимости от установленного уровня логирования, который загружается из конфигурационного файла settings.json через SettingsManager.

При возникновении событий в системе вызывается метод ObserveService.raise\_event, передающий информацию о событии в зарегистрированные наблюдатели, включая Logging. Метод handle\_event в классе Logging определяет, нужно ли записывать сообщение в лог в зависимости от его уровня, а метод \_log\_event форматирует и записывает его в файл application.log (рис. \ref{fig:logs}). Вся обработка логирования включает проверку необходимости записи события, его форматирование с добавлением временной метки и запись в файл.

\myfigure{0.9}{logs}{Отрывок лог-файла application.log}{logs}

Логирование в системе автоматизированного нормоконтроля охватывает различные типы событий (рис. \ref{fig:observe_logs}). Можно создать информационную запись, фиксирующую выполнение операции. Также возможно записать отладочное сообщение, помогающее анализировать процесс взаимодействия с системой. В случае возникновения ошибки система регистрирует соответствующую запись, что позволяет оперативно выявлять и устранять проблемы.

\myfigure{0.999}{observe_logs}{Примеры вызова логирования в системе}{observe_logs}

Итак, логирование в системе автоматизированного нормоконтроля обеспечивает контроль за работой приложения, помогает в диагностике и анализе его работы, а также упрощает процесс поиска и устранения ошибок. Гибкая настройка уровней логирования позволяет фильтровать записываемые события в зависимости от потребностей пользователя и этапа разработки системы.

\section{Разработка Telegram-бота и CLI для пользователей системы}

В качестве способов взаимодействия с системой автоматического нормоконтроля реализовано два интерфейса: Telegram-бот и CLI (Command Line Interface). Это позволяет пользователям выбирать наиболее удобный для них способ работы: через привычный мессенджер или через терминал, в зависимости от условий и предпочтений.

Telegram-бот представляет собой программного агента, с которым пользователь взаимодействует через чат. Такой формат является интуитивно понятным и привычным для большинства современных пользователей, так как не требует установки дополнительных приложений и легко доступен с любого устройства, где установлен мессенджер Telegram. Одним из преимуществ Telegram-ботов также является простота и быстрота их разработки: существуют готовые библиотеки и подробная документация, а основная логика строится на обработке текстовых команд и сообщений.

Создание Telegram-бота начинается с регистрации через сервис BotFather, официального бота Telegram для управления другими ботами. Пользователь задаёт название и уникальное имя для нового бота, после чего получает токен доступа, необходимый для программной работы с Telegram API. Далее разработчик использует этот токен в своей программе для настройки обработки запросов, отправки сообщений и выполнения логики взаимодействия. Также с помощью BotFather было установлено изображение в качестве \break аватара (рис. \ref{fig:bot}).

\myfigure{0.6}{bot}{Скриншот c информацией о Telegram-боте}{bot}

Telegram-бот реализован как удобный способ взаимодействия пользователей с системой автоматического нормоконтроля. Он позволяет студентам и нормоконтролёрам работать с системой через привычный интерфейс мессенджера, обеспечивая доступ только к тем функциям, которые соответствуют их роли. После запуска бот задаёт пользователю роль студента и предлагает доступный набор команд. Чтобы получить роль нормоконтролёра, нужно ввести секретный код, и тогда, помимо команд доступных студенту, будут доступны команды изменения правил. Для выполнения запросов, таких как проверка документа или изменение правил, бот обращается к серверному FastAPI-сервису системы нормоконтроля, отправляя соответствующие HTTP-запросы и обрабатывая полученные ответы. Это позволяет централизованно выполнять всю логику на стороне основного сервиса.

%Файловая структура Telegram-бота

Структура Telegram-бота организована в виде отдельного репозитория, в котором размещены все необходимые модули и файлы, обеспечивающие его работу (рис. \ref{fig:telega_struct}). Точкой входа является файл bot.py, где происходит запуск и конфигурация бота. Настройки, такие как токен, URL\_API и переменные окружения, вынесены в файл config.py и хранятся в отдельном .env файле. Для организации логирования используется модуль logger.py, а данные о ролях пользователей сохраняются в roles.db через модуль db.py. Структура разделена по функциональным блокам: обработчики команд сгруппированы в папке handlers, где start.py содержит команды начальной инициализации и вспомогательные команды, documents.py —-- команды для работы с документами, а rules.py —-- команды для получения и изменения правил оформления. Обращение к серверному FastAPI-сервису реализовано в модуле services/api.py, что позволяет централизованно обрабатывать все запросы к API. Дополнительно репозиторий включает лог-файл bot.log, документацию в виде README.md, файл зависимостей requirements.txt, а также .gitignore для исключения временных и конфиденциальных файлов из системы контроля версий.

\myfigure{0.8}{telega_struct}{Файловая структура Telegram-бота}{telega_struct}

В файле bot.py инициализируются основные компоненты бота: создаются экземпляры Bot и Dispatcher из библиотеки aiogram, подгружается токен из конфигурационного файла, инициализируется база данных для хранения ролей пользователей и подключаются обработчики команд. Все модули, отвечающие за обработку сообщений (start, documents, rules), регистрируются в диспетчере, а затем запускается постоянный опрос Telegram-серверов методом run\_polling. При этом осуществляется логирование запуска бота и возможных критических ошибок.

Файл config.py отвечает за загрузку конфигурации окружения. С помощью библиотеки dotenv из .env файла подгружаются такие переменные, как токен бота, секретный код для назначения роли проверяющего, URL-адрес к API системы нормоконтроля и USER\_ID администратора.

Модуль db.py реализует работу с базой данных roles.db, в которой хранятся данные о пользователях Telegram и их ролях (рис. \ref{fig:db}). При первом запуске инициализируется таблица, если она ещё не существует. При обращении к боту вызывается функция get\_user\_role, которая либо возвращает роль уже зарегистрированного пользователя, обновляя при необходимости время последней активности, либо добавляет нового пользователя с ролью студента. Также реализована функция set\_user\_role, позволяющая изменить роль пользователя, которая вызывается, например,при вводе секретного кода. Вся работа с базой построена на библиотеке sqlite3 не требующей отдельного сервера баз данных.

\myfigure{0.9}{db}{Таблица базы данных пользователей бота}{db}

Модуль logger.py настраивает логирование работы бота. Логгер записывает события в файл bot.log, используя формат, включающий дату, уровень события и сообщение. Также добавлен пользовательский уровень логирования FEEDBACK, предназначенный для записи обратной связи от пользователей. Такой подход к логированию помогает отслеживать как стандартные действия, так и пользовательские сообщения, позволяя оперативно реагировать на ошибки и предложения.

%Модуль базовых команд и управления ролями

Модуль handlers/start.py отвечает за обработку базовых команд, доступных пользователю после запуска Telegram-бота, а также за управление ролями. В начале файла создаётся объект Router, к которому регистрируются хендлеры всех реализованных команд. Функция get\_available\_commands возвращает список команд, доступных пользователю в зависимости от его текущей роли — студенту предлагаются команды для просмотра правил и проверки документов, а нормоконтролёру дополнительно доступны функции редактирования правил и сброса своей роли.

Команда /start запускает сессию и отображает пользователю приветственное сообщение с текущей ролью и списком доступных команд. Команда /set\_reviewer позволяет студенту изменить свою роль на нормоконтролёра при вводе корректного секретного кода, который сравнивается с переменной из файла настроек. В случае успешного ввода происходит смена роли и отображается обновлённый список команд (рис. \ref{fig:bot_start}). Если код неверен, пользователю выводится сообщение об ошибке, а попытка фиксируется в логах.

\myfigure{0.99}{bot_start}{Доступные команды студенту и нормоконтролеру}{bot_start}

Команда /my\_role сообщает текущую роль пользователя, а /help выводит перечень всех команд, доступных в соответствии с этой ролью. Команда /info предоставляет подробную информацию о назначении и возможностях бота, описывая, какие типы файлов он обрабатывает и какие элементы проверяет в каждом формате. Команда /reset\_role используется нормоконтролёрами для возврата к роли студента. Это действие фиксируется в логах и сразу отражается в интерфейсе пользователя. Команда /feedback позволяет отправить отзыв или замечание. Текст отзыва логируется с уровнем FEEDBACK и автоматически пересылается администратору в Telegram (рис. \ref{fig:feedback}). 

\myfigure{0.95}{feedback}{Уведомление и логирование при отправке отзыва пользователем}{feedback}

Файл api.py содержит набор функций для взаимодействия Telegram-бота с серверным API системы нормоконтролера. Он выполняет HTTP-запросы к различным эндпоинтам, обрабатывает ответы и возвращает результат. Через функцию get\_doc\_options бот получает список доступных типов документов, а get\_rules используется для получения правил оформления по заданному типу документа. Для нормоконтролёров предусмотрены функции change\_rule и change\_rule\_for\_all, которые позволяют изменять конкретное правило либо сразу для всех типов документов.

Проверка документов реализована двумя функциями: validate\_docx\_document отправляет на сервер .docx файл для анализа, а validate\_latex\_document — пару файлов .tex и .sty. В случае успешного ответа возвращаются результаты в формате JSON, при ошибках информация логируется и пользователю возвращается соответствующее сообщение об ошибке. Таким образом, api.py выступает связующим звеном между ботом и серверной частью системы, обеспечивая корректную передачу данных и устойчивость к сетевым сбоям.

Файл rules.py реализует обработку Telegram-команд, связанных с правилами оформления документов, и обращается к серверу через функции, импортированные из api.py. Он позволяет пользователям запрашивать доступные типы документов, просматривать правила по конкретному типу, а также, если у пользователя есть роль нормоконтролёра, изменять отдельные правила или применять изменения сразу ко всем типам. Через FSMContext реализуется пошаговое взаимодействие, если пользователь не указал тип документа сразу. Проверки прав доступа и обработка команд выполняются с логированием. Тип документа может быть введён пользователем в любом виде: строчными или заглавными буквами, с пробелами, подчёркиваниями или в смешанном регистре — всё это автоматически нормализуется в нужный формат, чтобы избежать ошибок при распознавании и повысить удобство работы с ботом.

Файл documents.py реализует обработку команд /check\_docx и /check\_latex. Для каждого формата создаются свои FSM-группы состояний: DocxCheck и LatexCheck. Для DOCX реализовано два состояния: ожидание файла и ожидание типа документа. Для LaTeX — три: ожидание .tex, .sty и типа. Состояния хранятся не более 5 минут. Также реализована функция check\_file\_size, ограничивающая размер файла до 25 МБ. Команды /check\_docx и /check\_latex могут принимать тип документа как аргумент или же запрашивать его после получения файлов. Если тип указан сразу, бот переходит к ожиданию файла, если пользователь не указал тип документа сразу, то бот запросит его позже отдельно.

Файлы обрабатываются поэтапно: сначала загружаются и проверяются расширения, затем сохраняются во временное хранилище FSM, после чего вызываются функции validate\_docx\_document или validate\_latex\_document. Эти функции возвращают результат в формате dict, который форматируется в понятное сообщение для пользователя через format\_validation\_result (рис. \ref{fig:bot_result}). Для LaTeX дополнительно реализован вспомогательный метод process\_latex\_validation, который объединяет все собранные данные (.tex, .sty, doc\_type) и вызывает проверку. После каждого завершения проверки состояние сбрасывается.

\myfigure{0.9}{bot_result}{Сообщение от бота с результатом проверки документа}{bot_result}

%% Про CLI %%

CLI (Command Line Interface, интерфейс командной строки) —-- это способ взаимодействия пользователя с программным обеспечением посредством ввода текстовых команд через терминал или командную строку. Такой интерфейс позволяет быстро выполнять операции без запуска графического интерфейса, веб-сайта или Telegram-бота. Более того, преимуществом CLI является возможность работы без доступа в интернет. Так, становится возможной офлайн-проверка документов в условиях ограниченного подключения или при необходимости локального использования.

Для системы автоматического нормоконтроля был разработан CLI-интерфейс, предоставляющий доступ ко всем основным функциям. Функционал CLI аналогичен списку запросов из API системы. Реализация CLI выполнена в виде модуля cli.py с использованием стандартной библиотеки argparse, предназначенной для парсинга аргументов командной строки.

CLI состоит из набора команд, каждая из которых соответствует определённой функции системы. Команда list-doc-types выводит список доступных типов документов. Команда get-rules <doc\_type> позволяет получить правила оформления для указанного типа работы. Команды update-rule и update-rule-all позволяют обновить конкретное правило либо для одного типа документа, либо сразу для всех. Команды validate-docx и validate-latex запускают процесс проверки структуры и оформления документов форматов DOCX и LaTeX соответственно. Входные данные, такие как пути к документам или параметры команд, передаются через аргументы после написания команды, а результат работы системы выводится в человекочитаемом формате JSON. При возникновении ошибок CLI предоставляет понятные сообщения, указывающие на источник проблемы. Чтобы нужная функция выполнилась, в командной строке необходимо написать, например, <<python cli.py \break validate-docx C:\textbackslash path\textbackslash to\textbackslash file.docx diploma>>.

Таким образом, CLI-интерфейс делает систему более универсальной и гибкой, предоставляя дополнительный способ взаимодействия. CLI можно запускать из любого терминала, предварительно настроив окружение.

%\section{Настройка автоматического развертывания с помощью GitHub Actions}

\section{Тестирование системы} % продолжить про тестирование на реале

Тестирование системы автоматического нормоконтроля было осуществлено с помощью автоматизированных юнит-тестов и тестирования на реальных работах студентов с помощью Telegram-бота. Юнит-тестирование проводилось для различных частей системы с целью проверки корректности работы основных модулей и компонентов. Тестирование на реальных работах проводилось с целью проверки работы и Telegram-бота, и качества проверки документов, и понимания пользователями, как пользоваться системой.

Были разработаны тесты для менеджера настроек (test\_settings.py), обеспечивающего загрузку, сохранение и обработку конфигурационных параметров системы. Проверена работа механизма загрузки настроек из файла, корректность обработки ошибок при открытии несуществующих или повреждённых файлов, а также функционирование паттерна Singleton при создании экземпляров менеджера.

Тестирование охватывало сервисы, отвечающие за обработку документов и проверку правил (test\_services.py). Была проверена корректность загрузки правил из файлов, работа механизма получения доступных типов правил и документов, а также целостность данных, используемых в процессе нормоконтроля.

Также были протестированы основные модели данных (test\_models.py), включая документы, ошибки, рекомендации и правила. Проверялась их корректная инициализация, преобразование в словарное представление и восстановление из него, а также сравнение экземпляров моделей между собой. Дополнительно тестировались строковые представления объектов для удобства отладки и логирования.

Кроме того, написаны тесты для проверки логики работы парсинга и валидации соответствия правилам оформления для LaTeX (test\_latex\_logics.py) и DOCX (test\_docx\_logics.py).

Дополнительно для проверки корректности работы API (test\_api.py) были разработаны тесты, использующие TestClient из библиотеки FastAPI. Тестирование охватывало ключевые запросы к системе, включая получение доступных типов документов и правил, обновление правил и проверку документов.

Результаты тестирования показали, что ключевые компоненты системы функционируют в соответствии с заданными требованиями. Автоматизированные тесты позволили выявить и исправить ряд ошибок, связанных с обработкой данных и их валидацией, что повысило стабильность и надёжность системы.

\section{Технико-экономическое обоснование проекта}

В этом разделе представлена информация, необходимая для подробного анализа всех финансовых и трудовых ресурсов, которые будут задействованы в процессе разработки и внедрения системы автоматизации нормоконтроля. Более того, учитываются не только прямые затраты, связанные с оплатой труда участников проекта, но и косвенные расходы, включая страховые взносы, дополнительные выплаты и накладные расходы. Оценка затрат позволяет более точно прогнозировать бюджет проекта.
Для оценки затрат на реализацию проекта был произведен расчет трудозатрат по каждому этапу, видам работ и исполнителям (прил. В). Наиболее трудоемкими этапами являются <<Проектирование системы>> и <<Разработка системы>>, что обусловлено необходимостью детальной проработки архитектуры, интерфейсов и функционала системы. Эти этапы потребуют наибольшего количества времени, что отражается в суммарных затратах, составляющих 100 200 и 148 200 рублей соответственно. 
Расходы на тестирование системы также составляют значительную часть, что связано с проведением разных видов тестирования, необходимых для проверки работоспособности системы. 
Кроме того, можно отметить, что работы по анализу аналогов и (прил. Б) требований, а также по проектированию системы выполняются как студентом-разработчиком, так и консультантом. Это позволяет задействовать внешнего эксперта для уточнения специфических требований и анализа нормативной базы.
В результате расчет общей суммы заработной платы составляет 425 200 рублей.

Для оценки финансовой стороны проекта также важно рассчитать себестоимость внедрения проекта, в которой указаны все статьи затрат, включая основную зарплату, дополнительные выплаты, страховые взносы, прочие прямые затраты и накладные расходы (табл. \ref{table:table23}).

\renewcommand{\arraystretch}{1}
\begin{table}[H]
    \centering   % располагаем таблицу по центру
    \caption{Расчет себестоимости внедрения проекта}    % название таблицы
    \begin{tabular}{|m{6.9cm}|m{4.4cm}|m{4cm}|}        % описываем 3 столбца таблицы
    \hline   % горизонтальная черта
    \centering\arraybackslash Статьи затрат & \centering\arraybackslash Сумма затрат, руб. & \centering\arraybackslash Удельный вес, \% \\ \hline
    1. Основная зарплата & 382 800 & 60,68 \\ \hline
    2. Дополнительная зарплата & 42400 & 6,72 \\ \hline
    3. Страховые взносы & 141282 & 22,40 \\ \hline
    4. Прочие прямые затраты & 7000 & 1,11 \\ \hline
    5. Накладные расходы & 57348 & 9,09 \\ \hline
    Итого & 630830 & 100,00 \\ \hline
    \end{tabular}
    \label{table:table23}
\end{table}

Основная часть затрат (60,68\%) составляет основная зарплата, учитывая значительное количество трудозатрат на всех этапах разработки. Дополнительная зарплата и страховые взносы составляют 6,72\% и 22,40\% от общей суммы затрат соответственно, что отражает оплату консультаций. Прочие прямые затраты, составляющие 1,11\%, охватывают траты на электроэнергию. Накладные расходы (9,09\%) включают в себя все остальные косвенные затраты, такие как амортизация оборудования и аренда серверов. Общая сумма затрат на внедрение проекта составляет 630 830 рублей, что полностью покрывает все финансовые расходы на разработку, внедрение и эксплуатацию системы.

Далее рассмотрены потенциальные доходы, которые могут быть получены в результате внедрения и монетизации системы автоматизации нормоконтроля. Оценка доходов позволяет оценить финансовую отдачу от проекта, проанализировать его прибыльность и сформировать прогноз на будущие периоды. 
В расчете доходов от реализации программного обеспечения в течение 6 лет указано количество проданных копий программного обеспечения, цена каждой копии, выручка от продаж, текущие затраты, прибыль от реализации и чистая прибыль после налога на прибыль (прил. Г). По истечении шести периодов наблюдается стабильный рост выручки и прибыли от реализации программного обеспечения. За этот период выручка увеличивается с 500 000 рублей в первом периоде до 1 276 282 рублей в шестом, что свидетельствует о высоком темпе роста продаж. Прибыль от реализации также увеличивается на протяжении всех периодов, достигая 1 148 653 рублей в конце шестого периода. Чистая прибыль, после учета налога на прибыль, составляет 300 000 рублей в первом периоде и 861 490 рублей в шестом. Таким образом, проект показывает устойчивый рост доходности и прибыльности, что подтверждает его финансовую успешность.

Также проект предусматривает получение доходов от лицензирования дополнительных модулей и консультационных услуг. В совокупности ожидаемые годовые доходы составляют 1 286 348 рублей, что подтверждает финансовую устойчивость и долгосрочную перспективу проекта.

Для оценки эффективности проекта «Система автоматизации нормоконтроля» используются методы анализа денежных потоков и экономических показателей. Оценка позволяет определить, насколько выгодным является проект в долгосрочной перспективе, учитывая как начальные инвестиции, так и будущие доходы и расходы. В данном разделе представлены расчет денежных потоков и итоговые экономические показатели проекта, которые позволят получить полную картину его финансовой состоятельности.
Для анализа финансовой эффективности проекта представлен расчет денежных потоков проекта на протяжении шести периодов (рис. \ref{fig:cashflow}). 

\myfigure{0.9}{cashflow}{Расчет денежных потоков проекта}{cashflow}

В первый год проект столкнется с отрицательным денежным потоком, так как инвестиционные затраты составляют 630 830 рублей, а эффект от внедрения системы только начинает проявляться. Однако со второго года проект начинает приносить прибыль, и денежный поток становится положительным, увеличиваясь каждый год. Например, в шестом году денежный поток составляет 1 148 653 рубля. Дисконтированные денежные потоки учитывают временную стоимость денег и показывают, что проект начнет приносить чистую прибыль с учетом дисконтирования уже во втором году. К концу шестого года накопленный дисконтированный денежный поток составляет 1 660 142 рубля.

Для получения общей оценки проекта были рассчитаны итоговые экономические показатели, которые говорят о его целесообразности (табл. \ref{table:table21}).

\renewcommand{\arraystretch}{1} 
\newcolumntype{C}{ >{\centering\arraybackslash} m{5cm} }  % так можно создать новый тип столбца с вертикальным центрированием (m), чтобы не дублировать длинную запись (указать нужную ширину в сантиметрах). Применяется, например, в команде begin
\begin{table}[H]
    \centering   % располагаем таблицу по центру
    \caption{Итоговые экономические показатели проекта}    % название таблицы
    \begin{tabular}{|m{8.89cm}|m{4.4cm}|m{2.4cm}|}        % описываем 3 столбца таблицы
    \hline   % горизонтальная черта
    \centering\arraybackslash Наименование показателя & \centering\arraybackslash Единица измерения & \centering\arraybackslash Значение \\ \hline
    Дисконтированный срок окупаемости & год & 2 \\ \hline
    Чистый дисконтированный доход (NPV) & руб. & 104713 \\ \hline
    Внутренняя норма доходности (IRR) & \% & 88 \\ \hline
    \end{tabular}
    \label{table:table21}
\end{table}

%%%%%% обновить гит когда спросишь про оформление этой таблицы %%%%%

Дисконтированный срок окупаемости проекта составляет всего 2 года, что является отличным результатом. Чистый дисконтированный доход (NPV) проекта составляет 104 713 рублей, что подтверждает его прибыльность с учетом всех затрат и дисконтирования. Внутренняя норма доходности (IRR) проекта равна 88\%, что является высоким показателем и свидетельствует о рентабельности инвестиций в проект.

Оценка эффективности проекта «Система автоматизации нормоконтроля» показывает его высокую финансовую привлекательность. Проект быстро окупается и обеспечивает положительные денежные потоки уже со второго года реализации. С учётом высоких значений экономических показателей проект является жизнеспособным и выгодным для инвестирования.

\section*{Выводы по главе}
\addcontentsline{toc}{section}{Выводы по главе}

В данной главе был рассмотрен процесс разработки системы автоматического нормоконтроля. Требования к структуре и оформлению документов оформлены в формате JSON-файлов, которые обслуживаются сервисом RuleService, что позволяет управлять правилами. Реализована проверка структуры студенческих работ в форматах LaTeX с использованием классов LatexParser и LatexChecker, которые анализируют документ и выявляют несоответствия установленным требованиям. В документах формата DOCX проверка структуры студенческих работ реализована в рамках проверки глав с помощью классов DocxParser, DocxChecker.Проверка оформления в LaTeX осуществляется через сравнение пользовательского sty-файла с эталонным, а в DOCX проверяется размер шрифта глав. Более того, разработано API, обеспечивающее взаимодействие системы с внешними сервисами, предоставляющее возможность проверять документы, просматривать и обновлять правила. Кроме того, в систему внедрено логирование, фиксирующее события и ошибки. Вместе с тем, проведено модульное тестирование системы с помощью юнит-тестов, по результатам которого внесены корректировки в систему. В результате создана основа для автоматизированного нормоконтроля, обладающая перспективой дальнейшего развития.


\chapter*{ЗАКЛЮЧЕНИЕ}
\addcontentsline{toc}{chapter}{ЗАКЛЮЧЕНИЕ} % это будет отображаться в содержании
В ходе выполнения данной работы была разработана система автоматического нормоконтроля, предназначенная для проверки оформления студенческих работ в форматах DOCX и LaTeX. В начале были проанализированы существующие решения и потребности потенциальных пользователей. Далее была продумана концепция системы и спроектирована система автоматического нормоконтроля, предоставляющая открытый доступ студентам к предварительной проверке своих работ. В ходе проектирования системы были разработаны различные диаграммы, которые структурированно описали архитектуру и функциональность будущей системы. Также проведён обзор инструментов разработки, в результате которого были выбраны технологии, такие как Python, Flask, dedoc, Git и Docker. Затем была продумана структура python-проекта и реализованы модели данных. Далее был реализован модуль управления правилами, организованными в формате JSON файла. Важным этапом работы была частичная реализация модулей проверки структуры и оформления студенческих работ формата Latex и DOCX. Реализация API системы стала ещё одним значимым этапом работы. Более того, в системе был разработан модуль логирования. К тому же, были написаны юнит-тесты для проведения модульного тестирования системы.

%%%%%%%%%%%%% ПЕРЕЧИТАТЬ И ДОПОЛНИТЬ ЗАКЛЮЧЕНИЕ %%%%%%%%%%%%

Технико-экономическое обоснование говорит о рентабельности проекта, целесообразности его внедрения и быстрой окупаемости.

Таким образом, выполнены все поставленные задачи и достигнута цель
работы.




\chapter*{СПИСОК ИСПОЛЬЗОВАННЫХ ИСТОЧНИКОВ}
\addcontentsline{toc}{chapter}{СПИСОК ИСПОЛЬЗОВАННЫХ ИСТОЧНИКОВ} % это будет отображаться в содержании


\begin{thebibliography}{}

\bibitem{serv1} 
Antiplagiat Killer. Оформить работу по ГОСТу онлайн бесплатно : [Электронный ресурс]. --– URL: https://killer-antiplagiat.ru/oformit-rabotu-po-gostu-onlajn-besplatno (дата обращения: 17.02.2025).
    
\bibitem{serv2}
ДокСтандарт. Автоматическое форматирование учебных \break работ : [Электронный ресурс]. --– URL: https://dokstandart.ru/  (дата  обращения: 17.02.2025).

\bibitem{serv3}
Uwd. Оформление студенческих работ. 4 шага! : [Электронный ресурс]. --– URL: https://uwd.su/oformlenie/ (дата обращения: 17.02.2025).


\bibitem{dedoc}
Dedoc: the system for document structure extraction. Документация : [Электронный ресурс]. --– URL: https://dedoc.readthedocs.io/en/latest/index.html (дата обращения: 28.02.2025).

\bibitem{fastapi}
FastAPI. Документация : [Электронный ресурс]. --– URL: https://fastapi.tiangolo.com/ru/ (дата обращения: 28.02.2025).


\end{thebibliography}

\newpage
%\chapter*{ПРИЛОЖЕНИЯ}
\addcontentsline{toc}{chapter}{ПРИЛОЖЕНИЯ}
% \begin{flushright}
%      {\bf Приложение А}
% \end{flushright}

% \begin{center} {\bf Название приложения} \end{center}
\addcontentsline{toc}{section}{Приложение А Ежедневные записи студента по практике}
\includepdf[pages={1}]{TitlePages/ezhe.pdf}
% Ежедневные записи


\newpage

\begin{flushright}
     { \bf Приложение Б}
\end{flushright}

\begin{center}  {\bf Репозитории системы} \end{center}
\addcontentsline{toc}{section}{Приложение Б Репозитории системы}

https://github.com/DocumentCheckingAutomatisation/document-checker --- репозиторий прототипа системы нормоконтроля

https://github.com/DocumentCheckingAutomatisation/tg-bot-doccheck --- репозиторий Telegram-бота

\newpage

\begin{flushright}
     { \bf Приложение В}
\end{flushright}

\begin{center}  {\bf Расчет трудозатрат на разработку системы} \end{center}
\addcontentsline{toc}{section}{Приложение В Расчет трудозатрат на разработку системы}

\begin{figure}[H]
\centering
    % width - отвечает за размер вашего рисунка
    % {kavichki.png} - это название рисунка из папки Images
    \includegraphics[width=1\linewidth]{spends.png}
    \captionsetup{justification=centering, format=plain}
    %\caption{Правильное использование кавычек} % Название рисунка
    %\label{fig:pic21}
\end{figure}

\newpage

\begin{flushright}
     { \bf Приложение Г}
\end{flushright}

\begin{center}  {\bf Выручка от реализации} \end{center}
\addcontentsline{toc}{section}{Приложение Г Выручка от реализации}

\begin{figure}[H]
\centering
    % width - отвечает за размер вашего рисунка
    % {kavichki.png} - это название рисунка из папки Images
    \includegraphics[width=1\linewidth]{finmarks.png}
    \captionsetup{justification=centering, format=plain}
    %\caption{Правильное использование кавычек} % Название рисунка
    %\label{fig:pic21}
\end{figure}

\end{document}

